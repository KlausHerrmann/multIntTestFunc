\documentclass[11pt]{article}
\usepackage[latin1,utf8]{inputenc}
\usepackage[T1]{fontenc}
\usepackage[english]{babel}
\usepackage{amsmath}
\usepackage{amsfonts}
\usepackage{amssymb}
\usepackage{bm}
\usepackage{dsfont} %\usepackage{bbm} might not work for double stroke 1
\usepackage{xcolor}
\usepackage{hyperref}
\usepackage{pdflscape}
\definecolor{links}{RGB}{0,0,128}

\hypersetup{
    unicode=true,          % non-Latin characters in Acrobat’s bookmarks
    pdftitle={},    % title
    pdfauthor={Klaus Herrmann},     % author
    colorlinks=true,       % false: boxed links; true: colored links
    linkcolor=links,          % color of internal links (change box color with linkbordercolor)
    citecolor=links,        % color of links to bibliography
    filecolor=links,      % color of file links
    urlcolor=links           % color of external links
}

\def\R{\mathbb{R}}
\def\Prob{\mathbb{P}}
\def\N{\mathbb{N}}
\def\bfx{\bm{x}}
\def\bfX{\bm{X}}
\def\bfv{\bm{v}}
\def\d{\,\mathrm{d}}
\newcommand{\abs}[1]{\left|#1\right|}
\definecolor{links}{RGB}{0,0,128}
\def\bfa{\bm{a}}
\def\bx{\mathbf{x}}
\def\ba{\mathbf{a}}
\def\bb{\mathbf{b}}
\def\Normal{\mathcal{N}}
\newcommand{\xNorm}[2]{\left \Vert #1 \right \Vert_{#2} }
\newcommand{\floor}[1]{\left \lfloor #1 \right \rfloor }
\newcommand{\ind}[1]{\mathds{1}_{#1} }

\begin{document}

\newpage
\title{\textbf{Mult}ivariate \textbf{Int}egration \textbf{Test Func}tions - Function References}
\author{Klaus Herrmann (Université de Sherbrooke)}
\maketitle

\tableofcontents

\section*{Introduction}
\addcontentsline{toc}{section}{Introduction}
This note compiles references for the integration test functions implemented in the R package \emph{multIntTestFunc}.
For each function there is either a derivation of the exact value of the integral or a reference to a derivation.
The functions are sorted by integration domains and Tables~\ref{table_Rn}--\ref{table_sphere} provide an overview over the test functions for the supported domains at the end of the document.
The available integration domains are
\begin{itemize}
	\item the non-negative real numbers $[0,\infty)^n$ (Section~P),
	\item the Euclidean space $\R^n$ (Section~R),
	\item the standard simplex $T_n$ (Section~T),
	\item the unit hypercube $C_n$ (Section~C),
	\item the unit ball $B_n$ (Section~B), and
	\item the unit sphere $S^{n-1}$ (Section~S).
\end{itemize}
For shorthand notation we use $\cdot$ for inner products, i.e., $\ba\cdot\bb = \sum_{i=1}^n a_ib_i$ for two vectors in $\R^n$.
%
%
%
%\newpage
%
%
%
\section*{P\hspace{0.5cm}Non-negative real numbers $[0,\infty)^n = \times_{i=1}^{n} [0,\infty)$}
\addcontentsline{toc}{section}{P\hspace{0.5cm}Non-negative real numbers $[0,\infty)^n = \times_{i=1}^{n} [0,\infty)$}
\subsection*{P.1\hspace{0.5cm}log-normal density}
\addcontentsline{toc}{subsection}{P.1\hspace{0.5cm}log-normal density}\label{lognormal}
Consider the function
\begin{align*}
f\colon [0,\infty)^n\to[0,\infty), \bx \mapsto f(\bx) =\frac{\exp(-((\ln(\bx)-\mu)^{T}\Sigma^{-1}(\ln(\bx)-\mu))/2)}{\prod_{i=1}^{n}x_i \sqrt{(2\pi)^n\det(\Sigma)}}.
\end{align*}
This is the density function of the log-normal distribution.
Therefore
\begin{align*}
\int_{[0,\infty)^n} f(\bx) \d \bx = 1.
\end{align*}
The $n$-dimensional vector $\mu$ and the symmetric positive definite matrix $\Sigma\in\R^{n\times n}$ are the mean and variance-covariance matrix of a multivariate normal random vector.
%
%
\subsection*{P.2\hspace{0.5cm}log-$t$ density}
\addcontentsline{toc}{subsection}{P.2\hspace{0.5cm}log-$t$ density}\label{logt}
Consider the function
\begin{align*}
f\colon &[0,\infty)^n\to[0,\infty),\\
&\bx \mapsto f(\bx) = \frac{(\prod_{i=1}^n x_i^{-1})\Gamma\left[(\nu+n)/2\right]}{\Gamma(\nu/2)\nu^{n/2}\pi^{n/2}\left|{\Sigma}\right|^{1/2}}\left[1+\frac{1}{\nu}({\log(\bx)}-{\delta})^{T}{\Sigma}^{-1}({\log(\bx)}-{\delta})\right]^{-(\nu+n)/2}
\end{align*}
This is the density function of the log-t distribution.
Therefore
\begin{align*}
\int_{[0,\infty)^n} f(\bx) \d \bx = 1.
\end{align*}
The $n$-dimensional vector $\delta$ and the symmetric positive definite matrix $\Sigma\in\R^{n\times n}$ are the location and scale matrix of a multivariate log-$t$ random vector.
The positive real number $\nu$ is the degrees-of-freedom parameter.
%
%
%
\section*{R\hspace{0.5cm}Euclidean Space $\R^n = \times_{i=1}^{n} \R$}
\addcontentsline{toc}{section}{R\hspace{0.5cm}Euclidean Space $\R^n = \times_{i=1}^{n} \R$}
\subsection*{R.1\hspace{0.5cm}Gaussian Integral}
\addcontentsline{toc}{subsection}{R.1\hspace{0.5cm}Gaussian Integral}
Consider the function
\begin{align*}
f\colon \R^n\to(0,\infty), \bx \mapsto f(\bx) = e^{-\xNorm{\bx}{2}^2}.
\end{align*}
As a standard result we have $\int_{\R} e^{-x^2}\d x = \pi^{1/2}$.
This leads to
\begin{align*}
\int_{\R^n} f(\bx) \d \bx
= \int_{\R^n} e^{-\xNorm{\bx}{2}^2} \d \bx
= \int_{\R^n} \prod_{i=1}^n e^{-x_i^2} \d \bx
= \prod_{i=1}^n \int_{\R^n}  e^{-x_i^2} \d x_i
= \pi^{n/2}.
\end{align*}
%
%
\subsection*{R.2\hspace{0.5cm}Floor Norm Integral}
\addcontentsline{toc}{subsection}{R.2\hspace{0.5cm}Floor Norm Integral}
For $s>1$ define the function
\begin{align*}
f\colon \R^n\to(0,\infty), \bx \mapsto f(\bx) = \frac{\Gamma(n/2+1)}{\pi^{n/2}(1+\lfloor \Vert \bx \Vert_2^n \rfloor)^s}.
\end{align*}
Denote by $B_{r}(0) = \{\bx\in\R^n : \xNorm{\bx}{2}<r\}$ the open ball centered at zero in $\R^n$.
In this case we have for $k=\{1,2,3,\ldots\}$ that
\begin{align*}
1+\lfloor \Vert \bx \Vert_2^n \rfloor = k \Leftrightarrow (k-1)^{1/n} \leq \xNorm{\bx}{2} < k^{1/n} \Leftrightarrow \bx \in B_{k^{1/n}}(0) \setminus B_{(k-1)^{1/n}}(0).
\end{align*}
Therefore $f$ is constant on the shells $S_k = B_{k^{1/n}}(0) \setminus B_{(k-1)^{1/n}}(0)$, where we set $S_1 = B_1(0)$, with value $f(\bx) = \Gamma(n/2+1)/(\pi^{n/2}k^s)$.
Therefore
\begin{align*}
\int_{\R^n} f(\bx) \d \bx
&= \int_{\uplus_{k\geq 1}S_k}  f(\bx) \d \bx
= \sum^{\infty}_{k=1} \int_{S_k}  f(\bx) \d \bx
= \sum^{\infty}_{k=1} \frac{\Gamma(n/2+1)}{\pi^{n/2}k^s} \text{vol}\left(S_k\right)\\
&= \sum^{\infty}_{k=1} \frac{\Gamma(n/2+1)}{\pi^{n/2}k^s} \left(\text{vol}(B_{k^{1/n}}(0))-\text{vol}(B_{(k-1)^{1/n}}(0))\right)\\
&= \sum^{\infty}_{k=1} \frac{\Gamma(n/2+1)}{\pi^{n/2}k^s} \left(\frac{\pi^{n/2}}{\Gamma(n/2+1)}(k^{1/n})^n-
\frac{\pi^{n/2}}{\Gamma(n/2+1)}((k-1)^{1/n})^n
\right)\\
&= \sum^{\infty}_{k=1} \frac{1}{k^s} = \zeta(s),
\end{align*}
where $\zeta$ is the Riemann zeta function.
%
%
%
\subsection*{R.3\hspace{0.5cm}Multivariate normal density}
\addcontentsline{toc}{subsection}{R.3\hspace{0.5cm}Multivariate normal density}
Consider the function
\begin{align*}
f\colon \R^n\to[0,\infty), \bx \mapsto f(\bx) =\frac{\exp(-((\bx-\mu)^{T}\Sigma^{-1}(\bx-\mu))/2)}{\sqrt{(2\pi)^n\det(\Sigma)}}.
\end{align*}
This is the density function of the multivariate normal distribution.
Therefore
\begin{align*}
\int_{\R^n} f(\bx) \d \bx = 1.
\end{align*}
The $n$-dimensional vector $\mu$ and the symmetric positive definite matrix $\Sigma\in\R^{n\times n}$ are the respective mean and variance-covariance matrix.
%
%
%
\subsection*{R.4\hspace{0.5cm}Multivariate $t$ density}
\addcontentsline{toc}{subsection}{R.4\hspace{0.5cm}Multivariate $t$ density}
Consider the function
\begin{align*}
f\colon \R^n\to[0,\infty), \bx \mapsto f(\bx) = \frac{\Gamma\left[(\nu+n)/2\right]}{\Gamma(\nu/2)\nu^{n/2}\pi^{n/2}\left|{\Sigma}\right|^{1/2}}\left[1+\frac{1}{\nu}({\bx}-{\delta})^{T}{\Sigma}^{-1}({\bx}-{\delta})\right]^{-(\nu+n)/2}.
\end{align*}
This is the density function of the multivariate $t$ distribution.
Therefore
\begin{align*}
\int_{\R^n} f(\bx) \d \bx = 1.
\end{align*}
The $n$-dimensional vector $\delta$ and the symmetric positive definite matrix $\Sigma\in\R^{n\times n}$ are the respective location vector and scale matrix.
The positive real number $\nu$ is the degrees-of-freedom parameter.
%
%
%
%\newpage
%
%
%
\section*{T\hspace{0.5cm}Standard Simplex $T_n = \{\bx\in\R^n : x_i \geq 0, \xNorm{\bx}{1} \leq 1\}$}
\addcontentsline{toc}{section}{T\hspace{0.5cm}Standard Simplex $T_n = \{\bx\in\R^n : x_i \geq 0, \xNorm{\bx}{1} \leq 1\}$}
For a continuously differentiable function $f \colon [0,1] \to \R$ we have, see \cite{IntegralSimplex}, that
\begin{align}\label{equation_simplex_sum}
\int_{T_n} f(x_1+\ldots+x_n) \d \bx = \frac{1}{\Gamma(n)} \int_0^1 f(s)s^{n-1}ds.
\end{align}
Equation~\eqref{equation_simplex_sum} can be used to construct integrable functions on $T_n$.
%
%
\subsection*{T.1\hspace{0.5cm}Dirichlet Integral}
\addcontentsline{toc}{subsection}{T.1\hspace{0.5cm}Dirichlet Integral}
For a vector $\bfv\in\R^{n+1}$ with strictly positive entries, i.e., $v_i>0$, define the function
\begin{align*}
f\colon T_{n}\to(0,\infty), \bx \mapsto f(\bx) = \prod_{i=1}^{n}x_i^{v_i-1}(1 - x_1 - \ldots - x_n)^{v_{n+1}-1}.
\end{align*}
It can be seen that the integral of $f$ over $T_n$ is the norming constant for the Dirichlet distribution.
We therefore have
\begin{align*}
\int_{T_n} f(\bx) \d\bx = \frac{\prod_{i=1}^{n+1}\Gamma(v_i)}{\Gamma(\sum_{i=1}^{n+1}v_i)}.
\end{align*}
%
%
\subsection*{T.2\hspace{0.5cm}Exponential of Sum}
\addcontentsline{toc}{subsection}{T.2\hspace{0.5cm}Exponential of Sum}
For a constant $c>0$ define the function
\begin{align*}
f\colon T_{n}\to(0,\infty), \bx \mapsto f(\bx) = e^{-c(x_1+\ldots+x_n)}.
\end{align*}
Combining \eqref{equation_simplex_sum} with integration-by-substitution yields
\begin{align*}
\int_{T_n} f(\bx) \d\bx
&= \frac{1}{\Gamma(n)} \int_0^1 e^{-cs}s^{n-1} \d s\\
&= \frac{c^{-n}}{\Gamma(n)} \int_0^c e^{-t}t^{n-1} \d t\\
&= \frac{\Gamma(n)-\Gamma(n,c)}{c^{n}\Gamma(n)},
\end{align*}
where $\Gamma(s,x)$ is the upper incomplete gamma function defined as
\begin{align*}
\Gamma(s,x) = \int_x^{\infty} t^{s-1}e^{-t} \d t.
\end{align*}
%
%
%
%\newpage
%
%
%
\section*{C\hspace{0.5cm}Unit Cube $C_n = [0,1]^n$}
\addcontentsline{toc}{section}{C\hspace{0.5cm}Unit Cube $C_n = [0,1]^n$}
\subsection*{C.1\hspace{0.5cm}Cosine Square}
\addcontentsline{toc}{subsection}{C.1\hspace{0.5cm}Cosine Square}
For a vector $\bfv\in\R^n$, $\bfv\neq \mathbf{0}_n$, define the function
\begin{align*}
f\colon C_{n}\to\R, \bx \mapsto f(\bx) = (\cos\left(\bfx\cdot\bfv\right))^2.
\end{align*}
Following \cite{IntegralCos2} one can use the identities $\cos(x)=\Re(e^{ix})$ and $\cos(x)^2 = \frac{1}{2}+\frac{1}{2}\cos(2x)$ to show
\begin{align*}
\int_{C_{n}} f(\bx) \d\bx =\frac{1}{2}+\frac{1}{2}\cos\left(\sum_{j=1}^{n}v_j\right)\prod_{j=1}^{n}\frac{\sin(v_j)}{v_j}.
\end{align*}
%
%
%
\subsection*{C.2\hspace{0.5cm}Floor of Sum}
\addcontentsline{toc}{subsection}{C.2\hspace{0.5cm}Floor of Sum}
For the function
\begin{align*}
f\colon C_{n}\to\R, \bx \mapsto f(\bx) = \lfloor x_1 + \ldots + x_n \rfloor
\end{align*}
we have, see \cite{IntegralFloorCube}, that
\begin{align*}
\int_{C_{n}} f(\bx) \d\bx = \frac{n-1}{2}.
\end{align*}
%
%
\subsection*{C.3\hspace{0.5cm}Maximum}
\addcontentsline{toc}{subsection}{C.3\hspace{0.5cm}Maximum}
For the function
\begin{align*}
f\colon C_{n}\to\R, \bx \mapsto f(\bx) = \max(x_1,\ldots,x_n)
\end{align*}
we have, see \cite{IntegralCubeMax}, that
\begin{align*}
\int_{C_{n}} f(\bx) \d\bx = \frac{n}{n+1}.
\end{align*}
%
%
\subsection*{C.4\hspace{0.5cm}Bratley-Fox-Niederreiter $I_4$}
\addcontentsline{toc}{subsection}{C.4\hspace{0.5cm}Bratley-Fox-Niederreiter $I_4$}
For the function
\begin{align*}
f\colon C_{n}\to\R, \bx \mapsto f(\bx) = \sum^{n}_{i=1} (-1)^i \prod_{j=1}^{i} x_j
\end{align*}
we have, see \cite{BratleyFoxNiederreiter1992}, that
\begin{align*}
\int_{C_{n}} f(\bx) \d\bx = -\frac{1}{3}\left(1-\left(\frac{1}{2}\right)^n\right).
\end{align*}
%
%
%
%\newpage
%
%
%
\section*{B\hspace{0.5cm}Unit Ball $B_n = \{\bx \in \mathbb{R}^n : \xNorm{\bx}{2} \leq 1\}$}
\addcontentsline{toc}{section}{B\hspace{0.5cm}Unit Ball $B_n = \{\bx \in \mathbb{R}^n : \xNorm{\bx}{2} \leq 1\}$}
%
%
%
\subsection*{B.1\hspace{0.5cm}Norm of Standard Normal Random Vector}
\addcontentsline{toc}{subsection}{B.1\hspace{0.5cm}Norm of Standard Normal Random Vector}
The function
\begin{align*}
f\colon B_{n}\to\R, \bx \mapsto f(\bx) = \frac{1}{(2\pi)^{n/2}}e^{-\xNorm{\bx}{2}^2/2}
\end{align*}
can be seen as the density of a random vector $\bfX=(X_1,\ldots,X_n)$, where all $X_i\sim\Normal(0,1)$ are independent.
We can then rewrite the integral as
\begin{align*}
\int_{B_{n}} f(\bx) \d\bx = \Prob[\xNorm{\bfX}{2}\leq 1] = \Prob[\xNorm{\bfX}{2}^2\leq 1] = F_{\chi^2_n}(1),
\end{align*}
where $F_{\chi^2_n}$ is the distribution function of a chi-square random variable.
%
%
%
\subsection*{B.2\hspace{0.5cm}Polynomials}
\addcontentsline{toc}{subsection}{B.2\hspace{0.5cm}Polynomials}
For non-negative integers $a_1,\ldots,a_n$, $a_i\in\{0,1,2,\ldots\}$, define the monomial
\begin{align*}
f\colon B_{n}\to\R, \bx \mapsto f(\bx) = \prod_{i=1}^{n}x_i^{a_i}.
\end{align*}
We know from \cite{Folland2001} that
\begin{align*}
\int_{B_{n}} f(\bx) \d\bx =
\begin{cases}
0, &\text{ if at least one $a_i$ is odd},\\
\frac{2\prod_{i=1}^{n}\Gamma(b_i)}{\Gamma(\sum^{n}_{i=1}b_i)(n+\sum_{i=1}^{n}a_i)}, b_i = \frac{a_i+1}{2}, &\text{ if all $a_i$ are even}.
\end{cases}
\end{align*}
%
%
%
%\newpage
%
%
\section*{S\hspace{0.5cm}Unit Sphere $S^{n-1} = \{\bx\in\R^n : \xNorm{\bx}{2} = 1\}$}
\addcontentsline{toc}{section}{S\hspace{0.5cm}Unit Sphere $S^{n-1} = \{\bx\in\R^n : \xNorm{\bx}{2} = 1\}$}
%
%
\subsection*{S.1\hspace{0.5cm}Inner Products}
\addcontentsline{toc}{subsection}{S.1\hspace{0.5cm}Inner Products}
For two vectors $\ba,\bb\in\R^n$ define the function
\begin{align*}
f\colon S^{n-1}\to\R, \bx \mapsto f(\bx) = (\ba\cdot\bx)(\bb\cdot\bx).
\end{align*}
From Proposition~2 in \cite{KhanPinsky2003} we know
\begin{align*}
\int_{S^{n-1}} f(\bx) \d\bx = \frac{\abs{S^{n-1}}}{n}(\ba\cdot\bb).
\end{align*}
For $n > 1$ we have $\abs{S^{n-1}} = \frac{2\pi^{n/2}}{\Gamma(n/2)}$.
For $n=1$ we have $S^{n-1} = \{-1,1\}$ and therefore the integral is zero.
%
%
%
\subsection*{S.2\hspace{0.5cm}Polynomials}
\addcontentsline{toc}{subsection}{S.2\hspace{0.5cm}Polynomials}
For non-negative integers $a_1,\ldots,a_n$, $a_i\in\{0,1,2,\ldots\}$, define the monomial
\begin{align*}
f\colon S^{n-1}\to\R, \bx \mapsto f(\bx) = \prod_{i=1}^{n}x_i^{a_i}.
\end{align*}
For $n >1$ we know from \cite{Folland2001} that
\begin{align*}
\int_{S^{n-1}} f(\bx) \d\bx =
\begin{cases}
0, &\text{ if at least one $a_i$ is odd},\\
\frac{2\prod_{i=1}^{n}\Gamma(b_i)}{\Gamma(\sum^{n}_{i=1}b_i)}, b_i = \frac{a_i+1}{2}, &\text{ if all $a_i$ are even}.
\end{cases}
\end{align*}
For $n=1$ we have $S^{n-1} = \{-1,1\}$ and therefore the integral is zero.
%
%
%
\section*{Tables of Integration Functions}
\addcontentsline{toc}{section}{Tables of Integration Functions}
The following Tables~\ref{table_Pn}--\ref{table_sphere} provide an overview over the available functions by integration domain.
The first column shows the S4 class name within the package, while the last column provides a reference to the relevant subsection in this document.
%
%
%
\begin{landscape}
\begin{table}
\center
\begin{tabular}{llllll}
%{p{2.0cm}p{3.0cm}p{3.0cm}p{4.0cm}}
\hline\hline
\rule{0pt}{3ex}
Pn\_& Parameters &  $f(\bfx)=$ & exact value & Properties & Details\\
\hline
\rule{0pt}{4ex}
lognormalDensity & $n,\in\N,\mu\in\R^n, \Sigma\in\R^{n\times n}$ & $\frac{\exp(-((\ln(\bx)-\mu)^{T}\Sigma^{-1}(\ln(\bx)-\mu))/2)}{\prod_{i=1}^{n}x_i \sqrt{(2\pi)^n\det(\Sigma)}}$ & $1.0$ & $C^{\infty}$ & P.1\\
\rule{0pt}{4ex}
logtDensity & $n,\in\N,\delta\in\R^n, \Sigma\in\R^{n\times n},\nu>0$ & (see P.2) & $1.0$ & $C^{\infty}$ & P.2\\
\hline\hline
\end{tabular}
\caption{Functions with integration domain $[0,\infty)^n$.}
\label{table_Pn}
\end{table}
\end{landscape}
%
%
%
%
%
\begin{landscape}
\begin{table}
\center
\begin{tabular}{llllll}
%{p{2.0cm}p{3.0cm}p{3.0cm}p{4.0cm}}
\hline\hline
\rule{0pt}{3ex}
Rn\_& Parameters &  $f(\bfx)=$ & exact value & Properties & Details\\
\hline
\rule{0pt}{4ex}
Gauss & $n \in \N$ & $\exp(-\xNorm{\bfx}{2}^2)$ & $\pi^{n/2}$ & $C^{\infty}$ & R.1\\
\rule{0pt}{4ex}
floorNorm & $n \in \N$, $s > 1$ & $\frac{\Gamma(n/2+1)}{\pi^{n/2}(1+\floor{\Vert \bfx \Vert_2^n})^s}$ & $\zeta(s)$ & non-continuous & R.2\\
normalDensity & $n,\in\N,\mu\in\R^n, \Sigma\in\R^{n\times n}$ & $\frac{\exp(-((\bx-\mu)^{T}\Sigma^{-1}(\bx-\mu))/2)}{\sqrt{(2\pi)^n\det(\Sigma)}}$ & $1.0$ & $C^{\infty}$ & R.3\\
tDensity & $n,\in\N,\delta\in\R^n, \Sigma\in\R^{n\times n}, \nu>0$ & (see R.4) & $1.0$ & $C^{\infty}$ & R.4\\
\hline\hline
\end{tabular}
\caption{Functions with integration domain $\R^n$.}
\label{table_Rn}
\end{table}
\end{landscape}
%
%
%
%
\begin{landscape}
\begin{table}
\center
\begin{tabular}{llllll}
%{p{2.0cm}p{3.0cm}p{3.0cm}p{4.0cm}}
\hline\hline
\rule{0pt}{3ex}
standardSimplex\_ & Parameters &  $f(\bfx)=$ & exact value & Properties & Details\\
\hline
\rule{0pt}{4ex}
Dirichlet & $n \in \N$, $\bfv \in (0,\infty)^{n+1}$ & $\prod_{i=1}^{n}x_i^{v_i-1}(1 - \sum^{n}_{i=1}x_i)^{v_{n+1}-1}$ & $\frac{\prod_{i=1}^{n+1}\Gamma(v_i)}{\Gamma(\sum_{i=1}^{n+1}v_i)}$ & $C^{\infty}$ & T.1\\
\rule{0pt}{4ex}
exp\_sum & $n \in \N$, $c>0$ & $\exp(-c(x_1+\ldots+x_n))$ & $\frac{\Gamma(n)-\Gamma(n,c)}{\Gamma(n)c^n}$ & $C^{\infty}$ & T.2\\
\hline\hline
\end{tabular}
\caption{Functions with integration domain $T_n$.}
\label{table_simplex}
\end{table}
\end{landscape}

\begin{landscape}
\begin{table}
\center
\begin{tabular}{llllll}
%{p{2.0cm}p{3.0cm}p{3.0cm}p{4.0cm}}
\hline\hline
\rule{0pt}{3ex}
unitCube\_ & Parameters &  $f(\bfx)=$ & exact value & Properties & Details\\
\hline
\rule{0pt}{4ex}
cos2 & $n \in \N$, $\bfv \in \R^n \setminus \mathbf{0}_n$ & $\left(\cos(\bfv\cdot\bfx)\right)^2$ & $\frac{1}{2}+\frac{1}{2}\cos(\bfv\cdot\mathbf{1}_n)\prod_{k=1}^{n}\frac{\sin(v_k)}{v_k}$ & $C^{\infty}$ & C.1\\
\rule{0pt}{4ex}
floor & $n \in \N$ & $\floor{x_1+\ldots+x_n}$ & $(n-1)/2$ & non-continuous & C.2\\\rule{0pt}{4ex}
max & $n \in \N$ & $\max(x_1,\ldots,x_n)$ & $n/(n+1)$ & continuous, non-differentiable & C.3\\\rule{0pt}{4ex}
BFN4 & $n \in \N$ & $\sum^{n}_{i=1} (-1)^i \prod_{j=1}^{i} x_j$ & $-(1-(1/2)^n)/3$ & $C^{\infty}$ & C.4\\
\hline\hline
\end{tabular}
\caption{Functions with integration domain $C_n=[0,1]^n$.}
\label{table_cube}
\end{table}
%
\end{landscape}








\begin{landscape}
\begin{table}
\center
\begin{tabular}{llllll}
%{p{2.0cm}p{3.0cm}p{3.0cm}p{4.0cm}}
\hline\hline
\rule{0pt}{3ex}
unitBall\_ & Parameters &  $f(\bfx)=$ & exact value & Properties & Details\\
\hline
\rule{0pt}{4ex}
normGauss & $n \in \N$ & $\frac{1}{(2\pi)^{n/2}}\exp(-\xNorm{\bfx}{2}^2/2)$ & $F_{\chi^2_n}(1)$ & $C^{\infty}$ & B.1\\
\rule{0pt}{4ex}
polynomial & $n \in \N$, $\bfa\in\{0,1,2,\ldots\}^n$ & $\prod_{i=1}^{n}x_i^{a_i}$ & (see details) & $C^{\infty}$ & B.2\\
\hline\hline
\end{tabular}
\caption{Functions with integration domain $B_n$.}
\label{table_ball}
\end{table}
\end{landscape}



\begin{landscape}
\begin{table}
\center
\begin{tabular}{llllll}
%{p{2.0cm}p{3.0cm}p{3.0cm}p{4.0cm}}
\hline\hline
\rule{0pt}{3ex}
unitSphere\_ & Parameters &  $f(\bfx)=$ & exact value & Properties & Details\\
\hline
\rule{0pt}{4ex}
innerProduct1 & $n \in \N$, $\ba,\bb\in\R^n$ & $(\ba\cdot\bx)(\bb\cdot\bx)$ & $\frac{\abs{S^{n-1}}}{n}(\ba\cdot\bb)$ & $C^{\infty}$ & S.1\\
\rule{0pt}{4ex}
polynomial & $n \in \N$, $\bfa\in\{0,1,2,\ldots\}^n$ & $\prod_{i=1}^{n}x_i^{a_i}$ & (see details) & $C^{\infty}$ & S.2\\
\hline\hline
\end{tabular}
\caption{Functions with integration domain $S^{n-1}$.}
\label{table_sphere}
\end{table}
\end{landscape}


\bibliography{bibDocumentationTestFunctions}{}
\bibliographystyle{plain}

%\begin{thebibliography}{100}  % 100 is a random guess of the total number of %references
%\bibitem{Folland2001}[Folland (2001)] Gerald B. Folland, How to Integrate a Polynomial over a Sphere. The American Mathematical Monthly, Vol. 108, No. 5 (May, 2001), pp. 446-448.
%\bibitem{KhanPinsky2003}[Khan and Pinsky (2003)] Khan, T., Pinsky, M. (2003). On some integrals over a unit sphere. Technical report.
%\end{thebibliography}

\end{document}

\begin{abstract}
%Third, the introduced risk measure is defined for random vectors instead of random variables.
\end{abstract}
%
%




\section{Numerical Integration Based on Quasi Random Numbers}
\subsection{Monte Carlo Integration for $A\subsetneq \R^n$}
To explain the code in the vignette it is beneficial to review Monte Carlo integration.
Monte Carlo integration is the simplest multivariate integration scheme and serves as a standard.
If we are interested in integrating a multivariate function $f$ over a domain $A \subsetneq \R^n$ we can resort to Monte Carlo integration if we can generate uniformly distributed (pseudo) random numbers over a set $B$ such that $A \subseteq B$ and $\leb(B)>0$.
\begin{align*}
\int_{A} f(\bfx) \d \bfx
&= \int_{B} f(\bfx) \ind{A}(\bfx) \frac{\leb(B)}{\leb(B)} \d \bfx
=  \leb(B)\int_{B} f(\bfx) \ind{A}(\bfx)  \frac{1}{\leb(B)}\d \bfx
= \leb(B) \E[f(\bfU) \ind{A}(\bfU)]\\
&\approx \leb(B) \frac{1}{n}\sum^{n}_{i=1} f(\bfU_i) \ind{A}(\bfU_i).
\end{align*}

\subsection{Integration over $\R^n$}
$\bfX$ with density $h(\bfx)>0$
\begin{align*}
\int_{\R^n} f(\bfx) \d \bfx
&= \int_{\R^n} f(\bfx) \frac{h(\bfx)}{h(\bfx)} \d \bfx\\
& = \int_{\R^n} \frac{f(\bfx)}{h(\bfx)} h(\bfx) \d \bfx\\
&= \E[f(\bfX)/h(\bfX)]\\
&\approx \frac{1}{n} \sum^{n}_{i=1} f(\bfX_i)/h(\bfX_i)
\end{align*}


%\bibliographystyle{apalike}
\bibliographystyle{plainnat}
\bibliography{references}
\end{document}
%% End:
